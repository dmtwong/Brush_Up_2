% Options for packages loaded elsewhere
\PassOptionsToPackage{unicode}{hyperref}
\PassOptionsToPackage{hyphens}{url}
%
\documentclass[
]{article}
\usepackage{amsmath,amssymb}
\usepackage{iftex}
\ifPDFTeX
  \usepackage[T1]{fontenc}
  \usepackage[utf8]{inputenc}
  \usepackage{textcomp} % provide euro and other symbols
\else % if luatex or xetex
  \usepackage{unicode-math} % this also loads fontspec
  \defaultfontfeatures{Scale=MatchLowercase}
  \defaultfontfeatures[\rmfamily]{Ligatures=TeX,Scale=1}
\fi
\usepackage{lmodern}
\ifPDFTeX\else
  % xetex/luatex font selection
\fi
% Use upquote if available, for straight quotes in verbatim environments
\IfFileExists{upquote.sty}{\usepackage{upquote}}{}
\IfFileExists{microtype.sty}{% use microtype if available
  \usepackage[]{microtype}
  \UseMicrotypeSet[protrusion]{basicmath} % disable protrusion for tt fonts
}{}
\makeatletter
\@ifundefined{KOMAClassName}{% if non-KOMA class
  \IfFileExists{parskip.sty}{%
    \usepackage{parskip}
  }{% else
    \setlength{\parindent}{0pt}
    \setlength{\parskip}{6pt plus 2pt minus 1pt}}
}{% if KOMA class
  \KOMAoptions{parskip=half}}
\makeatother
\usepackage{xcolor}
\usepackage[margin=1in]{geometry}
\usepackage{color}
\usepackage{fancyvrb}
\newcommand{\VerbBar}{|}
\newcommand{\VERB}{\Verb[commandchars=\\\{\}]}
\DefineVerbatimEnvironment{Highlighting}{Verbatim}{commandchars=\\\{\}}
% Add ',fontsize=\small' for more characters per line
\usepackage{framed}
\definecolor{shadecolor}{RGB}{248,248,248}
\newenvironment{Shaded}{\begin{snugshade}}{\end{snugshade}}
\newcommand{\AlertTok}[1]{\textcolor[rgb]{0.94,0.16,0.16}{#1}}
\newcommand{\AnnotationTok}[1]{\textcolor[rgb]{0.56,0.35,0.01}{\textbf{\textit{#1}}}}
\newcommand{\AttributeTok}[1]{\textcolor[rgb]{0.13,0.29,0.53}{#1}}
\newcommand{\BaseNTok}[1]{\textcolor[rgb]{0.00,0.00,0.81}{#1}}
\newcommand{\BuiltInTok}[1]{#1}
\newcommand{\CharTok}[1]{\textcolor[rgb]{0.31,0.60,0.02}{#1}}
\newcommand{\CommentTok}[1]{\textcolor[rgb]{0.56,0.35,0.01}{\textit{#1}}}
\newcommand{\CommentVarTok}[1]{\textcolor[rgb]{0.56,0.35,0.01}{\textbf{\textit{#1}}}}
\newcommand{\ConstantTok}[1]{\textcolor[rgb]{0.56,0.35,0.01}{#1}}
\newcommand{\ControlFlowTok}[1]{\textcolor[rgb]{0.13,0.29,0.53}{\textbf{#1}}}
\newcommand{\DataTypeTok}[1]{\textcolor[rgb]{0.13,0.29,0.53}{#1}}
\newcommand{\DecValTok}[1]{\textcolor[rgb]{0.00,0.00,0.81}{#1}}
\newcommand{\DocumentationTok}[1]{\textcolor[rgb]{0.56,0.35,0.01}{\textbf{\textit{#1}}}}
\newcommand{\ErrorTok}[1]{\textcolor[rgb]{0.64,0.00,0.00}{\textbf{#1}}}
\newcommand{\ExtensionTok}[1]{#1}
\newcommand{\FloatTok}[1]{\textcolor[rgb]{0.00,0.00,0.81}{#1}}
\newcommand{\FunctionTok}[1]{\textcolor[rgb]{0.13,0.29,0.53}{\textbf{#1}}}
\newcommand{\ImportTok}[1]{#1}
\newcommand{\InformationTok}[1]{\textcolor[rgb]{0.56,0.35,0.01}{\textbf{\textit{#1}}}}
\newcommand{\KeywordTok}[1]{\textcolor[rgb]{0.13,0.29,0.53}{\textbf{#1}}}
\newcommand{\NormalTok}[1]{#1}
\newcommand{\OperatorTok}[1]{\textcolor[rgb]{0.81,0.36,0.00}{\textbf{#1}}}
\newcommand{\OtherTok}[1]{\textcolor[rgb]{0.56,0.35,0.01}{#1}}
\newcommand{\PreprocessorTok}[1]{\textcolor[rgb]{0.56,0.35,0.01}{\textit{#1}}}
\newcommand{\RegionMarkerTok}[1]{#1}
\newcommand{\SpecialCharTok}[1]{\textcolor[rgb]{0.81,0.36,0.00}{\textbf{#1}}}
\newcommand{\SpecialStringTok}[1]{\textcolor[rgb]{0.31,0.60,0.02}{#1}}
\newcommand{\StringTok}[1]{\textcolor[rgb]{0.31,0.60,0.02}{#1}}
\newcommand{\VariableTok}[1]{\textcolor[rgb]{0.00,0.00,0.00}{#1}}
\newcommand{\VerbatimStringTok}[1]{\textcolor[rgb]{0.31,0.60,0.02}{#1}}
\newcommand{\WarningTok}[1]{\textcolor[rgb]{0.56,0.35,0.01}{\textbf{\textit{#1}}}}
\usepackage{graphicx}
\makeatletter
\def\maxwidth{\ifdim\Gin@nat@width>\linewidth\linewidth\else\Gin@nat@width\fi}
\def\maxheight{\ifdim\Gin@nat@height>\textheight\textheight\else\Gin@nat@height\fi}
\makeatother
% Scale images if necessary, so that they will not overflow the page
% margins by default, and it is still possible to overwrite the defaults
% using explicit options in \includegraphics[width, height, ...]{}
\setkeys{Gin}{width=\maxwidth,height=\maxheight,keepaspectratio}
% Set default figure placement to htbp
\makeatletter
\def\fps@figure{htbp}
\makeatother
\setlength{\emergencystretch}{3em} % prevent overfull lines
\providecommand{\tightlist}{%
  \setlength{\itemsep}{0pt}\setlength{\parskip}{0pt}}
\setcounter{secnumdepth}{-\maxdimen} % remove section numbering
\ifLuaTeX
  \usepackage{selnolig}  % disable illegal ligatures
\fi
\usepackage{bookmark}
\IfFileExists{xurl.sty}{\usepackage{xurl}}{} % add URL line breaks if available
\urlstyle{same}
\hypersetup{
  pdftitle={Markdown refresh},
  pdfauthor={David W.},
  hidelinks,
  pdfcreator={LaTeX via pandoc}}

\title{Markdown refresh}
\author{David W.}
\date{2024-06-09}

\begin{document}
\maketitle

\section{Section 1: C++ Introduction}\label{section-1-c-introduction}

\subsection{What is C++}\label{what-is-c}

C++ is a general purpose programming language that is free-form and
compiled. It is regarded as an intermediate-level language, as it
comprises both high-level and low-level language features. It provides
imperative, object-oriented and generic programming features.

The overall program has a structure, but it is also important to
understand the purpose of part of that structure.

\subsection{101 step by step}\label{step-by-step}

\begin{Shaded}
\begin{Highlighting}[]
\CommentTok{\#include\textless{}iostream\textgreater{}}

\NormalTok{using namespace std;}

\NormalTok{int }\FunctionTok{main}\NormalTok{()  \{}
    \SpecialCharTok{/}\ErrorTok{/}\NormalTok{ YOUR CODE GOES HERE}
    \SpecialCharTok{/}\ErrorTok{/}\NormalTok{ Please take input and print output to standard input}\SpecialCharTok{/}\FunctionTok{output}\NormalTok{ (stdin}\SpecialCharTok{/}\NormalTok{stdout)}
    \SpecialCharTok{/}\ErrorTok{/}\NormalTok{ E.g. }\StringTok{\textquotesingle{}cin\textquotesingle{}} \ControlFlowTok{for}\NormalTok{ input }\SpecialCharTok{\&} \StringTok{\textquotesingle{}cout\textquotesingle{}} \ControlFlowTok{for}\NormalTok{ output}
\NormalTok{    cout }\SpecialCharTok{\textless{}}\ErrorTok{\textless{}} \StringTok{"Hello, InterviewBit!"} \SpecialCharTok{\textless{}}\ErrorTok{\textless{}}\NormalTok{ endl;}
\NormalTok{    return }\DecValTok{0}\NormalTok{;}
\NormalTok{\}}
\end{Highlighting}
\end{Shaded}

\subsection{preprocessor}\label{preprocessor}

\begin{Shaded}
\begin{Highlighting}[]
\CommentTok{\#include\textless{}iostream\textgreater{}}
\end{Highlighting}
\end{Shaded}

\begin{itemize}
\item
  The hash sign (\#) signifies the start of a preprocessor command.
\item
  The \#include command is a specific preprocessor command that
  effectively copies and pastes the entire text of the file specified
  between the angle brackets into the source code. In this case, the
  file is ``iostream'' which is a standard file that should come with
  the C++ compiler. This file name is short for ``input-output
  streams''; in short, it contains code for displaying and getting the
  text from the user.
\item
  The include statement allows a programmer to ``include'' this
  functionality in the program without having to literally cut and paste
  it into the source code every time.
\item
  The iostream file is part of the C++ standard library, which provides
  a set of useful and commonly used functionality provided with the
  compiler. The ``include'' mechanism, however, can be used both for
  standard code provided by the compiler and for reusable files created
  by the programmer.
\end{itemize}

\subsection{namespace}\label{namespace}

\begin{Shaded}
\begin{Highlighting}[]
\NormalTok{using namespace std;}
\end{Highlighting}
\end{Shaded}

C++ supports the concept of namespaces. A namespace - is essentially a
prefix that is applied to all the names in a certain set. - like
toolboxes with different useful tools.

The using command tells the compiler to allow all the names in the
``std'' namespace to be usable without their prefix.

The iostream file defines three names used in this program - cout, cin,
and endl

Those three are all defined in the std namespace. ``std'' is short for
``standard'' since these names are defined in the standard C++ library
that comes with the compiler.

\textbf{Without using the std namespace, the names would have to include
the prefix and be written as std::cout, std::cin, and std::endl.}

\subsection{main}\label{main}

\begin{Shaded}
\begin{Highlighting}[]
\NormalTok{int }\FunctionTok{main}\NormalTok{()  \{}
\end{Highlighting}
\end{Shaded}

The starting point of all C++ programs is the main function. This
function is called by the operating system when your program is executed
by the computer.

\subsection{the three command in std
namespace}\label{the-three-command-in-std-namespace}

\begin{Shaded}
\begin{Highlighting}[]
\NormalTok{cout }\SpecialCharTok{\textless{}}\ErrorTok{\textless{}} \StringTok{"Hello, InterviewBit!"} \SpecialCharTok{\textless{}}\ErrorTok{\textless{}}\NormalTok{ endl;}
\end{Highlighting}
\end{Shaded}

\textbf{cin \& cout} - cout is character output - cin is character input

In a typical C++ program, most function calls are of the form
object.function\_name(argument1, argument2).

Symbols such as \textbf{\textless\textless{}} can also behave like
functions, as illustrated by the use of cout above. This capability is
called operator overloading.

\textbf{\{ \}} A block of code is defined with the \{ \} tokens.

\textbf{semicolons} Statements in C++ \textbf{must be terminated with a
semicolon.}

\subsection{return}\label{return}

The return keyword tells the program to return a value to the
\textbf{function int main} that called this function,\\
and then to continue execution in the int main function from the point
at which this function was called.

The \textbf{type of the value returned by a function} must match
\textbf{the type specified in the declaration} of the function.

\subsection{Common data types in
cpplus}\label{common-data-types-in-cpplus}

built-in data types are described as follows:

\begin{itemize}
\tightlist
\item
  Int (``\%d''): 32 Bit integer
\item
  Long (``\%ld''): 64 bit integer, from -9,223,372,036,854,775,808 to
  9,223,372,036,854,775,807
\item
  Char (``\%c''): Character type, can range only from -128 to 127
\item
  Boolean (either true or false)
\item
  Float (``\%f''): 32 bit real value
\item
  Double (``\%lf''): 64 bit real value
\end{itemize}

User Defined data types are - structures (struct) - classes (class).
This will be covered later

\subsection{set precision for float and
double}\label{set-precision-for-float-and-double}

To print float and double up to specific number of decimal places use:

cout\textless\textless std::fixed \textless\textless{}
std::setprecision(number of decimal places) - setprecision() is
available in the \textbf{iomanip} library.

\subsection{Type Modifiers}\label{type-modifiers}

The above types can be modified using the following type modifiers: -
signed - unsigned - short - long

\subsection{Typedefs}\label{typedefs}

Creating new names (i.e.~aliases) for existing types. Following is the
simple syntax to define a new type using typedef:

\begin{Shaded}
\begin{Highlighting}[]
\NormalTok{typedef int item}
\NormalTok{item number }\OtherTok{=} \DecValTok{0} \SpecialCharTok{/}\ErrorTok{/}\NormalTok{ number is a integer variable}
\end{Highlighting}
\end{Shaded}

As we have use typedef and created new name for int as item, item is
equivalent to int now

\subsection{Variables}\label{variables}

a variable is a container (storage area) to hold data.

int val = 10; val = 15;

\subsection{Exercise 1.2}\label{exercise-1.2}

Try the following example in the editorial below. You have to input 5
space-separated values: int, long, char, float and double respectively.

Print each value on a new line in the same order it is received as
input. Note that the floating point value should be correct up to 3
decimal places and the double to 9 decimal places.

Example Input:

5 1234567891231 z 24.23 1214523.028352

\begin{Shaded}
\begin{Highlighting}[]
\CommentTok{\#include\textless{}iostream\textgreater{}}
\CommentTok{\#include \textless{}iomanip\textgreater{} }
\NormalTok{using namespace std;}

\NormalTok{int }\FunctionTok{main}\NormalTok{()  \{}
\NormalTok{    int int\_1; long long\_1; char char\_1; float float\_1; double double\_1; }
\NormalTok{    cin }\SpecialCharTok{\textgreater{}}\ErrorTok{\textgreater{}}\NormalTok{ int\_1;}
\NormalTok{    cin }\SpecialCharTok{\textgreater{}}\ErrorTok{\textgreater{}}\NormalTok{ long\_1;   }
\NormalTok{    cin }\SpecialCharTok{\textgreater{}}\ErrorTok{\textgreater{}}\NormalTok{ char\_1;  }
\NormalTok{    cin }\SpecialCharTok{\textgreater{}}\ErrorTok{\textgreater{}}\NormalTok{ float\_1;  }
\NormalTok{    cin }\SpecialCharTok{\textgreater{}}\ErrorTok{\textgreater{}}\NormalTok{ double\_1;  }
    
\NormalTok{    cout }\SpecialCharTok{\textless{}}\ErrorTok{\textless{}}\NormalTok{ int\_1 }\SpecialCharTok{\textless{}}\ErrorTok{\textless{}}\NormalTok{ endl }\SpecialCharTok{\textless{}}\ErrorTok{\textless{}}\NormalTok{ long\_1 }\SpecialCharTok{\textless{}}\ErrorTok{\textless{}}\NormalTok{ endl }\SpecialCharTok{\textless{}}\ErrorTok{\textless{}}\NormalTok{ char\_1 }\SpecialCharTok{\textless{}}\ErrorTok{\textless{}}\NormalTok{ endl }\SpecialCharTok{\textless{}}\ErrorTok{\textless{}}\NormalTok{ fixed }\SpecialCharTok{\textless{}}\ErrorTok{\textless{}} \FunctionTok{setprecision}\NormalTok{(}\DecValTok{3}\NormalTok{) }\SpecialCharTok{\textless{}}\ErrorTok{\textless{}}\NormalTok{ float\_1 }\SpecialCharTok{\textless{}}\ErrorTok{\textless{}}\NormalTok{ endl }\SpecialCharTok{\textless{}}\ErrorTok{\textless{}}\NormalTok{ fixed }\SpecialCharTok{\textless{}}\ErrorTok{\textless{}} \FunctionTok{setprecision}\NormalTok{(}\DecValTok{9}\NormalTok{) }\SpecialCharTok{\textless{}}\ErrorTok{\textless{}}\NormalTok{ double\_1 }\SpecialCharTok{\textless{}}\ErrorTok{\textless{}}\NormalTok{ endl;}
        
\NormalTok{    return }\DecValTok{0}\NormalTok{;}
\NormalTok{\}}
\end{Highlighting}
\end{Shaded}

\subsection{Type Conversion}\label{type-conversion}

C++ allows us to convert data of one type to that of another. This is
known as type conversion.

There are two types of type conversion in C++.

Implicit Conversion Explicit Conversion (also known as Type Casting)

\subsection{Implicit Type Conversion}\label{implicit-type-conversion}

The type conversion that is done automatically done by the compiler is
known as implicit type conversion. This type of conversion is also known
as automatic conversion.

\begin{Shaded}
\begin{Highlighting}[]
\NormalTok{int a }\OtherTok{=} \DecValTok{10}\NormalTok{;}
\NormalTok{char b }\OtherTok{=} \StringTok{\textquotesingle{}A\textquotesingle{}}\NormalTok{;}
\NormalTok{a }\OtherTok{=}\NormalTok{ a }\SpecialCharTok{+}\NormalTok{ b; }\SpecialCharTok{/}\ErrorTok{/}\NormalTok{ y implicitly converted to int. ASCII value of }\StringTok{\textquotesingle{}A\textquotesingle{}}\NormalTok{ is }\DecValTok{65}
\NormalTok{cout}\SpecialCharTok{\textless{}}\ErrorTok{\textless{}}\NormalTok{a}\SpecialCharTok{\textless{}}\ErrorTok{\textless{}}\NormalTok{endl;}
\SpecialCharTok{/}\ErrorTok{/}\NormalTok{ value of a is }\DecValTok{75}\NormalTok{(}\DecValTok{10} \SpecialCharTok{+} \DecValTok{65}\NormalTok{)}

\NormalTok{int num\_int;}
\NormalTok{double num\_double }\OtherTok{=} \FloatTok{10.79}\NormalTok{;  }
\SpecialCharTok{/}\ErrorTok{/}\NormalTok{ implicit conversion  }
\SpecialCharTok{/}\ErrorTok{/}\NormalTok{ assigning a double value to an int variable  }
\NormalTok{num\_int }\OtherTok{=}\NormalTok{ num\_double;  }
\SpecialCharTok{/}\ErrorTok{/}\NormalTok{ Value of num\_int will be }\DecValTok{10}
\SpecialCharTok{/}\ErrorTok{/}\NormalTok{ Here, the double value is automatically converted to int by the compiler before it is assigned to the num\_int variable.}
\end{Highlighting}
\end{Shaded}

\subsection{Explicit Conversion}\label{explicit-conversion}

When the user manually changes data from one type to another, this is
known as explicit conversion. This type of conversion is also known as
type casting.

There are two major ways in which we can use explicit conversion in C++.
They are:

\begin{itemize}
\tightlist
\item
  C-style type casting (also known as cast notation)
\item
  type conversion operators
\end{itemize}

\subsection{C-style Type Casting}\label{c-style-type-casting}

defining the required type \textbf{in front of the expression in
parenthesis, i.e.~(data\_type)expression}. This can be also considered
as forceful casting.

(data\_type)expression;

\begin{Shaded}
\begin{Highlighting}[]
\NormalTok{int a }\OtherTok{=} \DecValTok{10}\NormalTok{;}
\NormalTok{char b }\OtherTok{=} \StringTok{\textquotesingle{}A\textquotesingle{}}\NormalTok{;}
\NormalTok{a }\OtherTok{=}\NormalTok{ a }\SpecialCharTok{+}\NormalTok{ (int)b;}
\NormalTok{cout}\SpecialCharTok{\textless{}}\ErrorTok{\textless{}}\NormalTok{a}\SpecialCharTok{\textless{}}\ErrorTok{\textless{}}\NormalTok{endl;}
\SpecialCharTok{/}\ErrorTok{/}\NormalTok{ value of a is }\DecValTok{75}
\end{Highlighting}
\end{Shaded}

\subsection{Type conversion operators}\label{type-conversion-operators}

C++ also has four operators for type conversion: - static\_cast:
static\_cast(variable) - dynamic\_cast - const\_cast - reinterpret\_cast

\begin{Shaded}
\begin{Highlighting}[]
\NormalTok{float f }\OtherTok{=} \FloatTok{4.5}\NormalTok{;   }
\SpecialCharTok{/}\ErrorTok{/}\NormalTok{ using cast operator }
\NormalTok{int b }\OtherTok{=}\NormalTok{ static\_cast}\SpecialCharTok{\textless{}}\NormalTok{int}\SpecialCharTok{\textgreater{}}\NormalTok{(f); }
\NormalTok{cout }\SpecialCharTok{\textless{}}\ErrorTok{\textless{}}\NormalTok{ b; }
\SpecialCharTok{/}\ErrorTok{/}\NormalTok{ value of b is }\DecValTok{4}
\NormalTok{To learn more about typecasting click http}\SpecialCharTok{:}\ErrorTok{//}\NormalTok{www.cplusplus.com}\SpecialCharTok{/}\NormalTok{doc}\SpecialCharTok{/}\NormalTok{tutorial}\SpecialCharTok{/}\NormalTok{typecasting}\SpecialCharTok{/}
\end{Highlighting}
\end{Shaded}

\subsection{Exercise 1.3}\label{exercise-1.3}

You are given a character called ch, print the ASCII value of the
character.

\begin{Shaded}
\begin{Highlighting}[]
\CommentTok{\#include\textless{}iostream\textgreater{}}

\NormalTok{using namespace std;}

\NormalTok{int }\FunctionTok{main}\NormalTok{()  \{}
\NormalTok{    char ch;}
\NormalTok{    cin}\SpecialCharTok{\textgreater{}}\ErrorTok{\textgreater{}}\NormalTok{ch;}
    \SpecialCharTok{/}\ErrorTok{/}\NormalTok{Your code goes here}
\NormalTok{    int a;}
\NormalTok{    a }\OtherTok{=}  \FunctionTok{int}\NormalTok{(ch);}
\NormalTok{    cout }\SpecialCharTok{\textless{}}\ErrorTok{\textless{}}\NormalTok{ a }\SpecialCharTok{\textless{}}\ErrorTok{\textless{}}\NormalTok{ endl;}
\NormalTok{    return }\DecValTok{0}\NormalTok{;}
\NormalTok{\}}
\end{Highlighting}
\end{Shaded}

\subsection{Math Library}\label{math-library}

The C++ math library is actually C's math library. It is easy to use and
is accessed by including cmath.

\begin{Shaded}
\begin{Highlighting}[]
\CommentTok{\#include \textless{}cmath\textgreater{}}
\end{Highlighting}
\end{Shaded}

Note: Trigonometric functions in cmath use RADIANS.

\subsection{Exercise 1.4}\label{exercise-1.4}

You are given two float variables A and B, perform the operations
defined in comments in the editor below.

\begin{Shaded}
\begin{Highlighting}[]
\CommentTok{\#include\textless{}iostream\textgreater{}}
\CommentTok{\#include\textless{}cmath\textgreater{}}
\NormalTok{using namespace std;}
\NormalTok{int }\FunctionTok{main}\NormalTok{()  \{}
\NormalTok{    float A }\OtherTok{=} \FloatTok{12.56}\NormalTok{, B }\OtherTok{=} \FloatTok{5.12}\NormalTok{;}
    \SpecialCharTok{/}\ErrorTok{/}\NormalTok{ Print the sum of cube of both A and B, and store it }\ControlFlowTok{in}\NormalTok{ float variable named }\StringTok{"cube\_val"}
\NormalTok{    float res\_1 }\OtherTok{=} \FunctionTok{pow}\NormalTok{(A,}\DecValTok{3}\NormalTok{) }\SpecialCharTok{+} \FunctionTok{pow}\NormalTok{(B,}\DecValTok{3}\NormalTok{);}
\NormalTok{    cout }\SpecialCharTok{\textless{}}\ErrorTok{\textless{}}\NormalTok{  res\_1 }\SpecialCharTok{\textless{}}\ErrorTok{\textless{}}\NormalTok{ endl;}
    
    \SpecialCharTok{/}\ErrorTok{/}\NormalTok{ Print the square root of cube\_val, and store it }\ControlFlowTok{in}\NormalTok{ float variable named }\StringTok{"sq\_val"}
\NormalTok{    float result\_2 }\OtherTok{=} \FunctionTok{sqrt}\NormalTok{(res\_1);}
\NormalTok{    cout }\SpecialCharTok{\textless{}}\ErrorTok{\textless{}}\NormalTok{ result\_2 }\SpecialCharTok{\textless{}}\ErrorTok{\textless{}}\NormalTok{ endl;}
    
    \SpecialCharTok{/}\ErrorTok{/}\NormalTok{ Print the sin of sq\_val}
\NormalTok{    cout }\SpecialCharTok{\textless{}}\ErrorTok{\textless{}} \FunctionTok{sin}\NormalTok{(result\_2) }\SpecialCharTok{\textless{}}\ErrorTok{\textless{}}\NormalTok{ endl;}
    
\NormalTok{    return }\DecValTok{0}\NormalTok{;}
\NormalTok{\}}
\end{Highlighting}
\end{Shaded}

\subsection{File Handling}\label{file-handling}

Files are used to store data in a storage device permanently. File
handling provides a mechanism to store the output of a program in a
file, and to perform various operations on it.

In C++, files are mainly dealt by using three classes \textbf{fstream,
ifstream, ofstream} available in the headerfile \textbf{fstream}.

\begin{itemize}
\tightlist
\item
  ofstream: Stream class, to write on files
\item
  ifstream: Stream class, to read from files
\item
  fstream: Stream class, to both read and write from/to files.
\end{itemize}

\subsection{Opening a file}\label{opening-a-file}

\begin{Shaded}
\begin{Highlighting}[]
\NormalTok{void }\FunctionTok{open}\NormalTok{(const char}\SpecialCharTok{*}\NormalTok{ file\_name,ios}\SpecialCharTok{::}\NormalTok{openmode mode);}
\end{Highlighting}
\end{Shaded}

The first argument of the open function \textbf{const char* file\_name)}
defines the name, and format of the file with the address of the file.
The second argument \textbf{ios::openmode mode} represents the mode in
which the file has to be opened.

\begin{itemize}
\tightlist
\item
  in: File open for reading: the internal stream buffer supports input
  operations.
\item
  out: File open for writing: the internal stream buffer supports output
  operations.
\item
  binary: Operations are performed in binary mode rather than text.
\item
  ate: The output position starts at the end of the file.
\item
  app: All output operations happen at the end of the file, appending to
  its existing contents.
\item
  trunc: Any contents that existed in the file before it is open are
  discarded.
\end{itemize}

\subsection{Default Open Modes:}\label{default-open-modes}

ifstream ios::in ofstream ios::out fstream ios::in \textbar{} ios::out
Note: You can combine two or more of these values by \textbar{} them
together.

\begin{Shaded}
\begin{Highlighting}[]
\NormalTok{ofstream outfile;}
\FunctionTok{outfile.open}\NormalTok{(}\StringTok{"file.dat"}\NormalTok{, ios}\SpecialCharTok{::}\NormalTok{out }\SpecialCharTok{|}\NormalTok{ ios}\SpecialCharTok{::}\NormalTok{trunc );}
\end{Highlighting}
\end{Shaded}

Example using ifstream \& ofstream classes.

\begin{Shaded}
\begin{Highlighting}[]
\CommentTok{\#include \textless{}iostream\textgreater{} }
\CommentTok{\#include \textless{}fstream\textgreater{} }
  
\NormalTok{using namespace std; }
  
\SpecialCharTok{/}\ErrorTok{/}\NormalTok{ Driver Code }
\NormalTok{int }\FunctionTok{main}\NormalTok{() }
\NormalTok{\{ }
\NormalTok{    ofstream fout; }\SpecialCharTok{/}\ErrorTok{/}\NormalTok{ Creation of ofstream class object }
\NormalTok{    string line; }
  
    \SpecialCharTok{/}\ErrorTok{/}\NormalTok{ by default ios}\SpecialCharTok{::}\NormalTok{out mode, automatically deletes }
    \SpecialCharTok{/}\ErrorTok{/}\NormalTok{ the content of file. To append the content, open }\ControlFlowTok{in}\NormalTok{ ios}\SpecialCharTok{:}\NormalTok{app }
    \SpecialCharTok{/}\ErrorTok{/} \FunctionTok{fout.open}\NormalTok{(}\StringTok{"sample.txt"}\NormalTok{, ios}\SpecialCharTok{::}\NormalTok{app) }
    \FunctionTok{fout.open}\NormalTok{(}\StringTok{"sample.txt"}\NormalTok{); }
  
    \SpecialCharTok{/}\ErrorTok{/}\NormalTok{ Execute a loop If file successfully opened }
    \ControlFlowTok{while}\NormalTok{ (fout) \{ }
        \FunctionTok{getline}\NormalTok{(cin, line);  }\SpecialCharTok{/}\ErrorTok{/}\NormalTok{ Read a Line from standard input }
  
        \ControlFlowTok{if}\NormalTok{ (line }\SpecialCharTok{==} \StringTok{"{-}1"}\NormalTok{) }\SpecialCharTok{/}\ErrorTok{/}\NormalTok{ Enter }\SpecialCharTok{{-}}\DecValTok{1}\NormalTok{ to exit }
            \ControlFlowTok{break}\NormalTok{; }
      
\NormalTok{        fout }\SpecialCharTok{\textless{}}\ErrorTok{\textless{}}\NormalTok{ line }\SpecialCharTok{\textless{}}\ErrorTok{\textless{}}\NormalTok{ endl; }\SpecialCharTok{/}\ErrorTok{/}\NormalTok{ Write line }\ControlFlowTok{in}\NormalTok{ file }
\NormalTok{    \} }
    \FunctionTok{fout.close}\NormalTok{(); }\SpecialCharTok{/}\ErrorTok{/}\NormalTok{ Close the File }
  
    
\NormalTok{    ifstream fin; }\SpecialCharTok{/}\ErrorTok{/}\NormalTok{ Creation of ifstream class object to read the file }
    \FunctionTok{fin.open}\NormalTok{(}\StringTok{"sample.txt"}\NormalTok{);  }\SpecialCharTok{/}\ErrorTok{/}\NormalTok{ by default open mode }\OtherTok{=}\NormalTok{ ios}\SpecialCharTok{::}\ControlFlowTok{in}\NormalTok{ mode }
  
    \SpecialCharTok{/}\ErrorTok{/}\NormalTok{ Execute a loop until }\FunctionTok{EOF}\NormalTok{ (End of File) }
    \ControlFlowTok{while}\NormalTok{ (fin) \{ }
        \FunctionTok{getline}\NormalTok{(fin, line); }\SpecialCharTok{/}\ErrorTok{/}\NormalTok{ Read a Line from File}
\NormalTok{        cout }\SpecialCharTok{\textless{}}\ErrorTok{\textless{}}\NormalTok{ line }\SpecialCharTok{\textless{}}\ErrorTok{\textless{}}\NormalTok{ endl;  }\SpecialCharTok{/}\ErrorTok{/}\NormalTok{ Print line }\ControlFlowTok{in}\NormalTok{ Console }
\NormalTok{    \} }
  
    \SpecialCharTok{/}\ErrorTok{/}\NormalTok{ Close the file }
    \FunctionTok{fin.close}\NormalTok{(); }
\NormalTok{    return }\DecValTok{0}\NormalTok{;}
\NormalTok{\}}
\end{Highlighting}
\end{Shaded}

\subsection{Random Numbers}\label{random-numbers}

\textbf{rand()} - rand() function is used to generate random numbers.

\begin{Shaded}
\begin{Highlighting}[]
\NormalTok{int }\FunctionTok{rand}\NormalTok{(void)}\SpecialCharTok{:} 
\NormalTok{cout}\SpecialCharTok{\textless{}}\ErrorTok{\textless{}}\FunctionTok{rand}\NormalTok{()}\SpecialCharTok{\textless{}}\ErrorTok{\textless{}}\NormalTok{endl; }\SpecialCharTok{/}\ErrorTok{/}\NormalTok{ Outputs any random number}
\end{Highlighting}
\end{Shaded}

returns a pseudo-random number in the range of 0 to RAND\_MAX.
RAND\_MAX: is a constant whose default value may vary between
implementations but it is granted to be at least 32767.

\textbf{srand()} sets the starting point for producing a series of
pseudo-random integers. - If srand() is not called, the rand() seed is
set as if srand(1) were called at program start.

\begin{Shaded}
\begin{Highlighting}[]
\NormalTok{void }\FunctionTok{srand}\NormalTok{( unsigned seed )}\SpecialCharTok{:} 
\end{Highlighting}
\end{Shaded}

Note: The pseudo-random number generator should only - be seeded once, -
before any calls to rand(), - and the start of the program.

It should not be - repeatedly seeded, or - reseeded every time you wish
to generate a new batch of pseudo-random numbers.

Standard practice is to use the result of a call to
\textbf{srand(time(0))} as the seed.

\begin{Shaded}
\begin{Highlighting}[]
\NormalTok{int }\FunctionTok{main}\NormalTok{()\{ }
    \SpecialCharTok{/}\ErrorTok{/}\NormalTok{ This program will create different sequence of  }
    \SpecialCharTok{/}\ErrorTok{/}\NormalTok{ random numbers on every program run  }
  
    \SpecialCharTok{/}\ErrorTok{/}\NormalTok{ Use current time as seed }\ControlFlowTok{for}\NormalTok{ random generator }
    \FunctionTok{srand}\NormalTok{(}\FunctionTok{time}\NormalTok{(}\DecValTok{0}\NormalTok{)); }
  
    \ControlFlowTok{for}\NormalTok{(int }\AttributeTok{i =} \DecValTok{0}\NormalTok{; i}\SpecialCharTok{\textless{}}\DecValTok{4}\NormalTok{; i}\SpecialCharTok{++}\NormalTok{) }
\NormalTok{        cout}\SpecialCharTok{\textless{}}\ErrorTok{\textless{}}\FunctionTok{rand}\NormalTok{()}\SpecialCharTok{\textless{}}\ErrorTok{\textless{}}\StringTok{" "}\NormalTok{; }
\NormalTok{    return }\DecValTok{0}\NormalTok{; }
\NormalTok{\} }
\end{Highlighting}
\end{Shaded}

\section{Section 2: Flow Control}\label{section-2-flow-control}

\subsection{Comparision Operation \&
If-Else}\label{comparision-operation-if-else}

\begin{itemize}
\item
  comparison operators such as ** ==, !=, \textgreater, \textless, **
  etc can be used in C++
\item
  These operators (cause the immediate statement in which they are
  contained to) return a Boolean value of either true or false.
\end{itemize}

Comparision Operator for all primitive data type (int, char, float,
bool, etc.): \textbf{== and != }

Comparision Operator for numeric data types only (int, float, double
etc.): \textbf{\textgreater, \textless, \textgreater=, \textless= }

\subsection{Conditional statements}\label{conditional-statements}

2 general types - if, else / else if, else - ``switch\ldots case''
construct

\subsection{Example 2.1:}\label{example-2.1}

Given an integer num denoting percentage of a student, calculate the
grade according to the below rules:

If num \textgreater= 90, grade A. If num \textgreater= 80, grade B. If
num \textgreater= 70, grade C. If num \textgreater= 60, grade D. If num
\textgreater= 50, grade E. Else grade will be F. Print a string
consisting of single character denoting the grade.

\begin{Shaded}
\begin{Highlighting}[]
\CommentTok{\#include\textless{}iostream\textgreater{}}

\NormalTok{using namespace std;}

\NormalTok{int }\FunctionTok{main}\NormalTok{()  \{}
\NormalTok{    int num;}
\NormalTok{    cin}\SpecialCharTok{\textgreater{}}\ErrorTok{\textgreater{}}\NormalTok{num;}
    \SpecialCharTok{/}\ErrorTok{/}\NormalTok{ Your code goes here}
\NormalTok{    char result;}
    \ControlFlowTok{if}\NormalTok{ (num }\SpecialCharTok{\textgreater{}=} \DecValTok{90}\NormalTok{)\{}
\NormalTok{        result }\OtherTok{=} \StringTok{\textquotesingle{}A\textquotesingle{}}\NormalTok{;}
    \SpecialCharTok{/}\ErrorTok{/}\FunctionTok{console.log}\NormalTok{(}\StringTok{"A"}\NormalTok{)}
\NormalTok{    \}}
    \ControlFlowTok{else} \ControlFlowTok{if}\NormalTok{ (num }\SpecialCharTok{\textgreater{}=} \DecValTok{80}\NormalTok{)\{}
\NormalTok{        result }\OtherTok{=} \StringTok{\textquotesingle{}B\textquotesingle{}}\NormalTok{;}
\NormalTok{    \}}
    \ControlFlowTok{else} \ControlFlowTok{if}\NormalTok{ (num }\SpecialCharTok{\textgreater{}=} \DecValTok{70}\NormalTok{)\{}
\NormalTok{        result }\OtherTok{=} \StringTok{\textquotesingle{}C\textquotesingle{}}\NormalTok{;}
\NormalTok{    \}}
    \ControlFlowTok{else} \ControlFlowTok{if}\NormalTok{ (num }\SpecialCharTok{\textgreater{}=} \DecValTok{60}\NormalTok{)\{}
\NormalTok{        result }\OtherTok{=} \StringTok{\textquotesingle{}D\textquotesingle{}}\NormalTok{;}
\NormalTok{    \}}
    \ControlFlowTok{else} \ControlFlowTok{if}\NormalTok{ (num }\SpecialCharTok{\textgreater{}=} \DecValTok{50}\NormalTok{)\{}
\NormalTok{        result }\OtherTok{=} \StringTok{\textquotesingle{}E\textquotesingle{}}\NormalTok{;}
\NormalTok{    \}}
    \ControlFlowTok{else}\NormalTok{ \{}
\NormalTok{        result }\OtherTok{=} \StringTok{\textquotesingle{}F\textquotesingle{}}\NormalTok{;}
\NormalTok{    \}}
\NormalTok{    cout }\SpecialCharTok{\textless{}}\ErrorTok{\textless{}}\NormalTok{ result }\SpecialCharTok{\textless{}}\ErrorTok{\textless{}}\NormalTok{ endl;}

\NormalTok{    return }\DecValTok{0}\NormalTok{;}
\NormalTok{\}}
\end{Highlighting}
\end{Shaded}

\subsection{Switch statement}\label{switch-statement}

\begin{Shaded}
\begin{Highlighting}[]
\ControlFlowTok{switch}\NormalTok{ (expression) \{}
\NormalTok{    case constant1}\SpecialCharTok{:}
        \ErrorTok{//}\NormalTok{ code to be executed }\ControlFlowTok{if}\NormalTok{ expression is equal to constant1;}
        \ControlFlowTok{break}\NormalTok{;}

\NormalTok{    case constant2}\SpecialCharTok{:}
        \ErrorTok{//}\NormalTok{ code to be executed }\ControlFlowTok{if}\NormalTok{ expression is equal to constant2;}
        \ControlFlowTok{break}\NormalTok{;}

\NormalTok{    default}\SpecialCharTok{:}
        \ErrorTok{//}\NormalTok{ code to be executed }\ControlFlowTok{if}
        \SpecialCharTok{/}\ErrorTok{/}\NormalTok{ expression doesn}\StringTok{\textquotesingle{}t match any constant}
\StringTok{\}}
\end{Highlighting}
\end{Shaded}

If there is a match, the corresponding code after the matching label is
executed until the break statement is encountered.

\begin{Shaded}
\begin{Highlighting}[]
\NormalTok{int x }\OtherTok{=} \DecValTok{2}\NormalTok{; }
\ControlFlowTok{switch}\NormalTok{ (x) }
\NormalTok{\{ }
\NormalTok{    case }\DecValTok{1}\SpecialCharTok{:} 
\NormalTok{        cout }\SpecialCharTok{\textless{}}\ErrorTok{\textless{}} \StringTok{"Choice is 1"}\NormalTok{; }
        \ControlFlowTok{break}\NormalTok{; }
\NormalTok{    case }\DecValTok{2}\SpecialCharTok{:} 
\NormalTok{        cout }\SpecialCharTok{\textless{}}\ErrorTok{\textless{}} \StringTok{"Choice is 2"}\NormalTok{; }
        \ControlFlowTok{break}\NormalTok{; }
\NormalTok{    case }\DecValTok{3}\SpecialCharTok{:} 
\NormalTok{        cout }\SpecialCharTok{\textless{}}\ErrorTok{\textless{}} \StringTok{"Choice is 3"}\NormalTok{; }
        \ControlFlowTok{break}\NormalTok{; }
\NormalTok{    default}\SpecialCharTok{:} 
\NormalTok{        cout }\SpecialCharTok{\textless{}}\ErrorTok{\textless{}} \StringTok{"Choice other than 1, 2 and 3"}\NormalTok{; }
        \ControlFlowTok{break}\NormalTok{; }
\NormalTok{\}}
\end{Highlighting}
\end{Shaded}

\subsection{Exercise 2.2}\label{exercise-2.2}

\begin{Shaded}
\begin{Highlighting}[]
\CommentTok{\#include\textless{}iostream\textgreater{}}

\NormalTok{using namespace std;}

\NormalTok{int }\FunctionTok{main}\NormalTok{()  \{}
\NormalTok{    int weekday;}
\NormalTok{    cin}\SpecialCharTok{\textgreater{}}\ErrorTok{\textgreater{}}\NormalTok{weekday;}
    \SpecialCharTok{/}\ErrorTok{/}\NormalTok{ YOUR CODE GOES HERE}
\NormalTok{    string str\_wkd;}
\NormalTok{    int numeric\_input }\OtherTok{=} \FunctionTok{int}\NormalTok{(weekday);}
    \ControlFlowTok{switch}\NormalTok{ (numeric\_input)\{}
\NormalTok{        case }\DecValTok{1}\SpecialCharTok{:} 
\NormalTok{            str\_wkd }\OtherTok{=} \StringTok{"Monday"}\NormalTok{;}
            \ControlFlowTok{break}\NormalTok{;}
\NormalTok{        case }\DecValTok{2}\SpecialCharTok{:}
\NormalTok{            str\_wkd }\OtherTok{=} \StringTok{"Tuesday"}\NormalTok{;}
            \ControlFlowTok{break}\NormalTok{;}
\NormalTok{        case }\DecValTok{3}\SpecialCharTok{:}
\NormalTok{            str\_wkd }\OtherTok{=} \StringTok{"Wednesday"}\NormalTok{;}
            \ControlFlowTok{break}\NormalTok{;}
\NormalTok{        case }\DecValTok{4}\SpecialCharTok{:}
\NormalTok{            str\_wkd }\OtherTok{=} \StringTok{"Thursday"}\NormalTok{;}
            \ControlFlowTok{break}\NormalTok{;}
\NormalTok{        case }\DecValTok{5}\SpecialCharTok{:} 
\NormalTok{            str\_wkd }\OtherTok{=} \StringTok{"Friday"}\NormalTok{;}
            \ControlFlowTok{break}\NormalTok{;}
\NormalTok{        case }\DecValTok{6}\SpecialCharTok{:}
\NormalTok{            str\_wkd }\OtherTok{=} \StringTok{"Saturday"}\NormalTok{;}
            \ControlFlowTok{break}\NormalTok{;}
\NormalTok{        case }\DecValTok{7}\SpecialCharTok{:}
\NormalTok{            str\_wkd }\OtherTok{=} \StringTok{"Sunday"}\NormalTok{;  }
            \ControlFlowTok{break}\NormalTok{;}
\NormalTok{    \}}
\NormalTok{    cout }\SpecialCharTok{\textless{}}\ErrorTok{\textless{}}\NormalTok{ str\_wkd }\SpecialCharTok{\textless{}}\ErrorTok{\textless{}}\NormalTok{ endl;}
\NormalTok{    return }\DecValTok{0}\NormalTok{;}
\NormalTok{\} }
\end{Highlighting}
\end{Shaded}

\subsection{Conditional or Ternary
Expression}\label{conditional-or-ternary-expression}

\begin{Shaded}
\begin{Highlighting}[]
\NormalTok{( condition ) ? expressionIfTrue }\SpecialCharTok{:}\NormalTok{ expressionIfFalse;}
\end{Highlighting}
\end{Shaded}

First the condition is evaluated, and the side effects of this
evaluation carry out their impact on the local environment.

If the result is true then only the \textbf{expressionIfTrue} is
evaluated (causing side effects) and this second result is the value of
the whole conditional expression, and the expressionIfFalse is not
evaluated (and hence cause no side effects).

If the condition evaluates to false, then the situation is converse

A common use of the conditional expression: - assign the value x or y to
a, depending on an easily decidable condition, say x \textgreater{} y.

\begin{Shaded}
\begin{Highlighting}[]
\NormalTok{int x }\OtherTok{=} \DecValTok{7}\NormalTok{;}
\NormalTok{int y }\OtherTok{=} \DecValTok{5}\NormalTok{;}
\NormalTok{int a }\OtherTok{=}\NormalTok{ ( x }\SpecialCharTok{\textgreater{}}\NormalTok{ y ) ? x }\SpecialCharTok{:}\NormalTok{ y;}
\NormalTok{cout }\SpecialCharTok{\textless{}}\ErrorTok{\textless{}}\NormalTok{ a }\SpecialCharTok{\textless{}}\ErrorTok{\textless{}}\NormalTok{ endl;}
\end{Highlighting}
\end{Shaded}

\subsection{Exercise 2.3}\label{exercise-2.3}

Given two integer x and y, using Conditional or Ternary expression,
print ``Robin'' if the value of x is less than or equal to y else print
``Rahul'' (without quotes).

\begin{Shaded}
\begin{Highlighting}[]
\CommentTok{\#include\textless{}iostream\textgreater{}}

\NormalTok{using namespace std;}

\NormalTok{int }\FunctionTok{main}\NormalTok{()  \{}
\NormalTok{    int x, y;}
\NormalTok{    cin}\SpecialCharTok{\textgreater{}}\ErrorTok{\textgreater{}}\NormalTok{x}\SpecialCharTok{\textgreater{}}\ErrorTok{\textgreater{}}\NormalTok{y;}
    \SpecialCharTok{/}\ErrorTok{/}\NormalTok{ YOUR CODE GOES HERE}
\NormalTok{    string ans }\OtherTok{=}\NormalTok{ ( x }\SpecialCharTok{\textless{}=}\NormalTok{ y )? }\StringTok{"Robin"} \SpecialCharTok{:} \StringTok{"Rahul"}\NormalTok{;}

\NormalTok{    cout }\SpecialCharTok{\textless{}}\ErrorTok{\textless{}}\NormalTok{ ans }\SpecialCharTok{\textless{}}\ErrorTok{\textless{}}\NormalTok{ endl;}

\NormalTok{    return }\DecValTok{0}\NormalTok{;}
\NormalTok{\}}
\end{Highlighting}
\end{Shaded}

\subsection{Loops}\label{loops}

3 Types of loop: 1. for 2. while 3. do\ldots while

\subsection{for loop}\label{for-loop}

for (initialization; condition; update) \{ // body of-loop \} for(int i
= 1; i \textless= 5; i++)\{ cout\textless\textless i\textless\textless''
``; \} 1. initialization - initializes variables and is executed only
once 2. condition - if true, the body of for loop is executed, if false,
the for loop is terminated 3. update - updates the value of initialized
variables and again checks the condition

\subsection{while loop}\label{while-loop}

while (condition) \{ statememt(s); \} int i = 1;

while(i \textless= 5)\{ cout\textless\textless i\textless\textless'' ``;
i++; \}

\subsection{do\ldots while loop}\label{dowhile-loop}

\begin{itemize}
\tightlist
\item
  do\ldots while loop checks the conditional statement after the first
  run, then continuing onto another iteration.
\item
  A do-while loop is used where your loop should execute at least one
  time. Ex: take an integer input from the user until the user has
  entered a positive number.
\end{itemize}

do \{ //body \} while (condition);

int i = 1; do \{ cout \textless\textless{} i \textless\textless{} '' ``;
i++; \} while(i \textless= 5); // the contition is being checked after
the first run

\subsection{Exercise 2.4}\label{exercise-2.4}

\begin{Shaded}
\begin{Highlighting}[]
\CommentTok{\#include\textless{}iostream\textgreater{}}

\NormalTok{using namespace std;}

\NormalTok{int }\FunctionTok{main}\NormalTok{()  \{}
\NormalTok{    int N;}
\NormalTok{    cin}\SpecialCharTok{\textgreater{}}\ErrorTok{\textgreater{}}\NormalTok{N;}
    \SpecialCharTok{/}\ErrorTok{/}\NormalTok{ YOUR CODE GOES HERE}
\NormalTok{    int i }\OtherTok{=} \DecValTok{0}\NormalTok{;}
    \ControlFlowTok{while}\NormalTok{( i }\SpecialCharTok{\textless{}=}\NormalTok{ N) \{}
        \ControlFlowTok{if}\NormalTok{(i \% }\DecValTok{2} \SpecialCharTok{!=} \DecValTok{0}\NormalTok{) \{}
\NormalTok{            cout }\SpecialCharTok{\textless{}}\ErrorTok{\textless{}}\NormalTok{ i }\SpecialCharTok{\textless{}}\ErrorTok{\textless{}}\NormalTok{ endl;}
\NormalTok{        \}}
\NormalTok{        i}\SpecialCharTok{++}\NormalTok{;}
\NormalTok{    \}}
\NormalTok{    return }\DecValTok{0}\NormalTok{;}
\NormalTok{\}}
\end{Highlighting}
\end{Shaded}

\subsection{Jump Statements}\label{jump-statements}

Jump statements are used to \textbf{interrupt the normal flow} of
program.

Types of Jump Statements: 1. break 2. continue 3. goto

\subsection{Break Statement}\label{break-statement}

\begin{itemize}
\tightlist
\item
  used inside loop, or switch statement.
\item
  compiler will abort the loop and continue to execute statements
  followed by loop. int a = 1;
\end{itemize}

while(a \textless= 10)\\
\{\\
if(a==5)\\
break;\\
a++;\\
\}\\
cout \textless\textless{} ``Value of a is''
\textless\textless a\textless\textless endl;\\
// Value of a is 5

\subsection{Continue Statement}\label{continue-statement}

\begin{itemize}
\tightlist
\item
  inside loop
\item
  skip the followling statements in the current loop, and resume the
  loop
\end{itemize}

int a = 0;\\
while(a \textless{} 10)\\
\{\\
a++;\\
if(a == 5)\\
continue;

\begin{verbatim}
cout << a << " ";  
\end{verbatim}

\}\\
// prints 1 2 3 4 6 7 8 9 10

\subsection{Goto Statement}\label{goto-statement}

\begin{itemize}
\tightlist
\item
  jumps from one point \textbf{to another point within a function}. goto
  label\_1; (a) \ldots{} .. \ldots{}\\
  \ldots{} .. \ldots{}\\
  \ldots{} .. \ldots{}\\
  label\_1: (b) statement;\\
  \ldots{} .. \ldots{}
\item
  When \textbf{goto label\_1;} (a) is encountered, the control of
  program jumps to \textbf{label\_1:} (b) and executes the code below
  it.
\end{itemize}

num = 10\\
if (num \% 2 == 0)\\
// jump to even\\
goto even;\\
else\\
// jump to odd\\
goto odd;\\
even:\\
cout \textless\textless{} num \textless\textless{} '' is even'';\\
// return if even\\
return;\\
odd:\\
cout \textless\textless{} num \textless\textless{} '' is odd'';

\subsection{Reason to Avoid goto
Statement}\label{reason-to-avoid-goto-statement}

\begin{itemize}
\tightlist
\item
  can write any C++ program without the use of goto statement
\item
  makes the logic of the program complex and tangled
\item
  a harmful construct and a bad programming practice
\end{itemize}

\subsection{Exercise 2.5}\label{exercise-2.5}

You are given an integer N, print all the odd values, for all i, where 0
\textless= i \textless{} N. Print each values on a seperate line. Note:
Use continue statement

\begin{Shaded}
\begin{Highlighting}[]
\CommentTok{\#include\textless{}iostream\textgreater{}}

\NormalTok{using namespace std;}

\NormalTok{int }\FunctionTok{main}\NormalTok{()  \{}
\NormalTok{    int N;}
\NormalTok{    cin}\SpecialCharTok{\textgreater{}}\ErrorTok{\textgreater{}}\NormalTok{N;}
    \SpecialCharTok{/}\ErrorTok{/}\NormalTok{ YOUR CODE GOES HERE}
    \ControlFlowTok{for}\NormalTok{(int }\AttributeTok{i =} \DecValTok{0}\NormalTok{; i }\SpecialCharTok{\textless{}=}\NormalTok{ N; i}\SpecialCharTok{++}\NormalTok{) \{}
        \ControlFlowTok{if}\NormalTok{ (i \% }\DecValTok{2} \SpecialCharTok{==} \DecValTok{0}\NormalTok{) \{}
\NormalTok{            continue;}
\NormalTok{        \}}
\NormalTok{        cout }\SpecialCharTok{\textless{}}\ErrorTok{\textless{}}\NormalTok{ i }\SpecialCharTok{\textless{}}\ErrorTok{\textless{}}\NormalTok{ endl;}
\NormalTok{    \}}
   
\NormalTok{    return }\DecValTok{0}\NormalTok{;}
\NormalTok{\}}
\end{Highlighting}
\end{Shaded}


\end{document}
